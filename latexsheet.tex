\documentclass[10pt,landscape]{article}
\usepackage{multicol}
\usepackage{calc}
\usepackage{ifthen}
\usepackage[landscape]{geometry}
\usepackage{hyperref}

% To make this come out properly in landscape mode, do one of the following
% 1.
%  pdflatex latexsheet.tex
%
% 2.
%  latex latexsheet.tex
%  dvips -P pdf  -t landscape latexsheet.dvi
%  ps2pdf latexsheet.ps


% If you're reading this, be prepared for confusion.  Making this was
% a learning experience for me, and it shows.  Much of the placement
% was hacked in; if you make it better, let me know...


% 2008-04
% Changed page margin code to use the geometry package. Also added code for
% conditional page margins, depending on paper size. Thanks to Uwe Ziegenhagen
% for the suggestions.

% 2006-08
% Made changes based on suggestions from Gene Cooperman. <gene at ccs.neu.edu>


% To Do:
% \listoffigures \listoftables
% \setcounter{secnumdepth}{0}


% This sets page margins to .5 inch if using letter paper, and to 1cm
% if using A4 paper. (This probably isn't strictly necessary.)
% If using another size paper, use default 1cm margins.
\ifthenelse{\lengthtest { \paperwidth = 11in}}
	{ \geometry{top=.25in,left=.25in,right=.25in,bottom=.25in} }
	{\ifthenelse{ \lengthtest{ \paperwidth = 297mm}}
		{\geometry{top=1cm,left=1cm,right=1cm,bottom=1cm} }
		{\geometry{top=1cm,left=1cm,right=1cm,bottom=1cm} }
	}

% Turn off header and footer
\pagestyle{empty}
 

% Redefine section commands to use less space
\makeatletter
\renewcommand{\section}{\@startsection{section}{1}{0mm}%
                                {-1ex plus -.5ex minus -.2ex}%
                                {0.5ex plus .2ex}%x
                                {\normalfont\large\bfseries}}
\renewcommand{\subsection}{\@startsection{subsection}{2}{0mm}%
                                {-1explus -.5ex minus -.2ex}%
                                {0.5ex plus .2ex}%
                                {\normalfont\normalsize\bfseries}}
\renewcommand{\subsubsection}{\@startsection{subsubsection}{3}{0mm}%
                                {-1ex plus -.5ex minus -.2ex}%
                                {1ex plus .2ex}%
                                {\normalfont\small\bfseries}}
\makeatother

% Define BibTeX command
\def\BibTeX{{\rm B\kern-.05em{\sc i\kern-.025em b}\kern-.08em
    T\kern-.1667em\lower.7ex\hbox{E}\kern-.125emX}}

% Don't print section numbers
\setcounter{secnumdepth}{0}


\setlength{\parindent}{0pt}
\setlength{\parskip}{0pt plus 0.5ex}


% -----------------------------------------------------------------------

\begin{document}

\raggedright
\footnotesize
\begin{multicols*}{3}


% multicol parameters
% These lengths are set only within the two main columns
%\setlength{\columnseprule}{0.25pt}
\setlength{\premulticols}{1pt}
\setlength{\postmulticols}{1pt}
\setlength{\multicolsep}{1pt}
\setlength{\columnsep}{2pt}
\section{Distributions}
\textbf{Discrete}
\begin{tabular}{@{}ll@{}}
\verb!Combinatorial! & Finite space, equally likely outcomes\\
\verb!! & $P(A)=card(A)/n$\\
\verb!Bernoulli! & Success probability of bent coin\\
\verb!!  & $p_0=q=(1-p)$ and $p_1=p$ \\
\verb!!  &$\mu=p, \sigma^2=pq$ \\
\verb!Binomial!& Probability of n heads in k tosses\\ 
\verb!! & $p_k={n \choose k} p^k(1-p)^{n-k}$\\
\verb!! &$\mu=np, \sigma^2=npq$\\
\verb!Geometric!  & Failures before 1st success, $k\geq 0$\\
\verb!! & $p_k= (1-p)^k p$ \\
\verb!! &$\mu=\frac{q}{p}, \sigma^2=\frac{q}{p^2}$\\
\verb!Negative Binomial!  & Failures before rth success $k\geq 0$\\
\verb!! & $w_r(k;p)={k+r-1 \choose k}q^kp^r$\\
\verb!! & Note the negative binomial is sum of\\ 
\verb!! & r \textbf{independent} geometric RVs.\\
\verb!! &$\mu=\frac{rq}{p}, \sigma^2=\frac{rq}{(p)^2}$\\
\verb!Poisson! & Characterization of rare events\\ 
\verb!! & $p_k = e^{-\lambda} \lambda^k / k! $\\
\verb!! &$\mu=\lambda, \sigma^2=\lambda$\\
\end{tabular}
\textbf{Continuous\\}
\begin{tabular}{@{}ll@{}}
	\verb!Uniform!& Agnostic about outcome on interval \\
	\verb!! & $f(x)=1/(b-a)$ on \\
	\verb!! &$ a<x<b$ and $0$ $o.w.$\\
	\verb!! &$\mu=\frac{1}{2}(a+b), \sigma^2=\frac{1}{12}(b-a)$\\
	\verb!Gamma!& Note $\Gamma =(n-1)!$  \\
	\verb!! &$g_n(t;\alpha)=\alpha \frac{(\alpha t)^{n-1}} {(n-1)!}e^{-\alpha t}$\\
	\verb!! &$\mu=\frac{1}{\alpha}  , \sigma^2=\frac{1}{\alpha^2}$\\
	\verb!Chi-squared!  & Finite $X_k ~\mathcal{N}(0,1)$; $V^2=\sum_{k=1}^{n} X_k^2$ \\
	\verb!! & $g_{n/2}(t;1/2)=\frac{1}{2^{n/2}\Gamma(\frac{n}{2})}t^{\frac{n}{2} -1}e^{-\frac{t}{2}}$\\
	\verb!! &$\mu=n  , \sigma^2=2n$\\	
	\verb!Exponential!  & A model for true randomness \\
	\verb!! & $f(x)=\alpha e^{-\alpha x}$ for $x>0$\\
	\verb!! &$\mu=\frac{1}{\alpha}  , \sigma^2=\frac{1}{\alpha^2}$\\
	\verb!Normal! & The one to rule them all... \\
	\verb!! & $f(x) = \frac{1}{\sqrt{(2 \pi)} \sigma} e^{-(x-\mu)^{2}/2 \sigma^{2}}$ \\
	\verb!! & \hspace{-75pt} Even function, strictly positive, dies monotonically...\\
	\verb!! & \hspace{-75pt} Can be sometimes written as $N(\mu,\sigma^2)$...\\
	\verb!Multivariate Normal! & Note that x and m are vectors \\
	\verb!! & \hspace{-55pt}$f(x)=\frac{1}{2\pi^{\frac{n}{2}} det(A)^{\frac{1}{2}}}exp(-\frac{1}{2}(x-m)A^{-1}(x-m)^T)$
	
\end{tabular}

\textbf{Negative Binomial Trick:}
$ { r+k-1 \choose k}q^kp^r={-r \choose k}(-q)^kp^r$
\section{Properties of Distributions}
\textbf{Shifting and Scaling} \textit{If two densities come from same type: $\frac{1}{a}f(\frac{x-m}{a})$ then $\mu =a\mu +m$ $\sigma^2=a^2\sigma^2$}\\
\textbf{Convolution }\textit{The convolution of any pair of distributions must necessarily be a distribution}\\
\textbf{Stability of the Normal}
\textit{Suppose $X_1,X_2,...X_n$ are independent and for each k $X_k$ is normal with mean $\mu_k$ and variance $\sigma_k$. Then $S_n = \sum_{k=1}^{n}X_k$ is normal with mean $\sum_{k=1}^{n} \mu_k$ and variance  $\sum_{k=1}^{n} \sigma_k$, or in other words normal densities are stable under convolution. Also stable under \textbf{non-degenerate} linear transformation}\\
\textbf{Translation Invariance: Uniform}\\
\textbf{Sum of Independent Poisson RVs}
\textit{For \textbf{independent} random variables drawn from Poisson distribution is Poisson and the mean is $\lambda_1 + \lambda_2$}\\
\textbf{Sum of Independent Binomials} is also binomial.\\
\textbf{Sum of Independent Gamma RVs}
\textit{Given $X_1 \sim g_{n_1}(x_1,\alpha)$ and $X_2 \sim g_{n_2}(x_2,\alpha)$ then $Z=X_1+X_2 \sim g_{n_1+n_2}(z,\alpha)$. Note the RV's must have the same mean $\frac{1}{\alpha}$}\\
\textbf{The marginals of a multivariate normal are normal }\textit{Note: a system of variables with marginal normal densities need not have a jointly normal density, and, in fact a joint density need not even exist}\\
\textbf{Sum of Exponentials} \textit{Sum of \textbf{independent} exponential random variables from a common exponential density $\alpha e^{-\alpha x}$ for $x>0$ is $S_n$, the sum $S_n= X_1+X_2+...+X_n$ has the \textbf{gamma density}}.\\
\textbf{Memoryless Property} \textit{\textbf{ONLY} the exponential and geometric distributions are memoryless. The random variable X exhibits the memoryless property if $P\{X>s+t|X>s\}=P\{X>t\}$ or recast
	$P\{X>s+t | X>s\}=P\{X>t\}$ if and only if $P\{X>s+t\}=P\{X>s\}P\{X>t\}$}\\
\textbf{Rotation of Bivariate Normal} \textit{There exists a rotation of a bivariate normal st. the normal coordinates become independent}
\textbf{Max variance for Bernoulli Trial} \textit{Is $1/4$ which occurs if $p=1/2$. Bernoulli trials are assumed to be independent unless otherwise stated.}\\
\textbf{Limit of the geometric} \textit{The exponential distribution is the continuous limit of the geometric distribution. Use two integration by parts to show variance is $\frac{1}{\alpha^2}$ and the mean is $\frac{1}{\alpha}$}\\
\textbf{The Gamma Density is Closed Under Convolutions} \textit{For every fixed $\alpha>0$ the family of gamma densities $\{g_v(\dot;\alpha), v>0\}$ is closed under convolutions, that is, $(g_\mu\star g_\nu)(x)=g_{\mu+\nu}(x) \forall\mu \nu$. Restatement of earlier property.}

\section{Expectation}
\textbf{Variance} $Var(X)=E(X^2)-E(X)^2$\\
\textbf{Covariance} $Cov(X,Y)=E(XY)- E(X)E(Y)$\\
\textbf{Law of the Unconcious Statistician}
 \textit{If X is a discrete r.v. and g is a function from {$R\rightarrow R$} then, $E(g(X))=\sum_{x}g(x)P(X=x)$}
\textbf{Independence} \textit{Independent variables are uncorrelated, but uncorrelated variables are not necessarily independent. However, this is a necessary and sufficient condition if we are talking about the marginals of a normal density}
\textbf{Additivity} \textit{Expectation is additive. Variance is additive if the summands are independent.}
\textbf{Conditional Expectation} \textit{$E(X_2)=E(E(X_2|X_1))$}\\


\section{Conditional Probability}
$P(A|H)=\frac{P(A \cap H)}{P(H)}$\\
$P_H: A \rightarrow P(A|H)$\\
$P(A_1 \cap A_2 \cap ... \cap A_n)= P(A_1|A_2 \cap ... \cap A_n)$\\ $\times P(A_2|A_3 \cap ... \cap A_n) \times P(A_{n-1}|P(A_n) \times P(A_n)$\\
\textbf{Bayes Rule}
$P(A_k|H)=\frac{P(H|A_k)P(A_k)}{\sum_{j \geq 1}{} P(H|A_j)P(A_j)}$

\section{Inclusion-Exclusion}
\textit{$S_k= \sum (P(A_{j_1}) \cap...\cap P(A_{j_k}))={n+1 \choose k} (1-ka)^n_+$}\\
\textbf{De Finneti's Theorem}\\
\textit{$P\{L_1>x_1,...,L_{n=1}>x_{n+1}\}=(1-\frac{x_1}{\tau}-\frac{x_2}{\tau}...-\frac{x_{n+1}}{\tau})^n_+$}\\
\section{Limit Laws}
\textbf{Weak Law of Large Numbers}\\
\textbf{Strong Law of Large Numbers}\\
\textbf{Central Limit Theorem}\\
\textit{Suppose $X_1,X-2,...$ is a sequence of independent random variable drawn from a common distribution F with mean zero and variance one. For each n, let $S_n*=(X_1+...+X_n)/\sqrt{n}$. Then $S_n*$ converges in distribution to the standard normal.}\\
	$P\{a<S_n*<b\}\rightarrow\Phi(b)-\Phi(a)=\frac{1}{\sqrt{2\pi}}\int_{a}^{b}e^{-x^2/2}dx$
\textit{If $E(X_k)=\mu$ and $Var(X_k)=\sigma^2$ then we can center and scale the variables such that $S_n*=(S_n - n\mu/\sqrt{n}\sigma)$ and the theorem holds for the new properly normalize $S_n*$.}

\section{Inequalities}
\textbf{Convexity} The chord lies above the curve\\
$\Psi(\alpha x +(1-\alpha)y)\leq \alpha\Psi(x)+(1-\alpha)\Psi(y)$\\
\textbf{Jensen} If $\Psi$ is convex , $X$ is integrable, and $\Psi(x)$ is integrable then:
$\Psi(E(x))\leq E(\Psi(x))$\\
\textbf{Another of Jensen} $E(X^2)\geq E(X)^2$\\
\textbf{AM-GM} $ x_1^{p_1} x_2^{p_2}...x_n^{p_n}\leq p_1 x_1 +p_2 x_2 + ... p_n x_n$\\
\textbf{Specialization of AM-GM} $x^{1/p} y^{1/q} \leq (1/p) x +(1/q) y$\\
simplifies to: $ xy \leq (1/p) x^p +(1/q) y^p$\\
\textbf{Holder's} $|E(XY)| \leq E(|XY|) \leq E(|X|)^{1/p} E(|Y|)^{1/q}$\\ 
\textbf{Cauchy-Schwarz} If $p=q=2$ in Holder's $\rightarrow$ $|E(XY)^2| \leq E(X^2)E(Y^2)$\\
\textbf{Minkowski} $p\geq 1$ then $ ||x+y|_p \leq ||x||_p + ||y||_p$
\textbf{Chebyshev}$P\{|S_n-n\mu|\geq n\epsilon\}\leq \frac{n\sigma^2}{n^2\epsilon^2}$\\
\textbf{Chernoff} $P\{S_n\geq t\} \leq (\inf\limits_{\lambda\geq 0} e^{-\lambda t/n}M(\lambda))^n $ and\\
$P\{S_n\leq t\} \leq (\inf\limits_{\lambda\geq 0} e^{-\lambda t/n}M(-\lambda))^n $\\ 
Where $M(\lambda)=E(e^{\lambda X_j})=\int_{\mathcal{R}} e^{\lambda x}dF(x)$ is the moment generating function\\
\textbf{Markov}$P\{X\geq a\} \leq \frac {E(X)}{a}$\\
\textbf{Normal Tail Bound p 317} $\frac{\phi(x)}{x}(1-\frac{1}{x^2}) < 1- \Phi(x) < \frac {\phi(x)}{x} $\\
\textbf{Normal Tail Bound p 165} $\int_{t}^{\infty}\phi(x) dx \leq \frac{1}{2}e^{-t^2/2}$
\section{Logarithmic Identities}
$y = \log _b \left( x \right){\rm{ iff }}x = b^y$\\
$\log _b \left( {\frac{x}{y}} \right) = \log _b \left( x \right) - \log _b \left( y \right) $\\
$\log _b \left( x \right) = \log _b \left( c \right)\log _c \left( x \right) = \frac{{\log _c \left( x \right)}}{{\log _c \left( b \right)}} $\\
$\log _b \left( {x^n } \right) = n\log _b \left( x \right) $\\
$\log _b \left( 1 \right) = 0 $\\
$\log _b \left( b \right) = 1 $\\
$\log _b \left( {xy} \right) = \log _b \left( x \right) + \log _b \left( y \right)$\\
\vspace{20pt}
\begin{multicols*}{3}
$log(\infty)\rightarrow \infty$\\
$log(0)\rightarrow -\infty$\\
$log(1)=0$
\end{multicols*}
\section{Exponential Identities}
\begin{multicols*}{2}
$x^a x^b = x^{\left( {a + b} \right)} $\\
$x^a y^a = \left( {xy} \right)^a$\\
$\left( {x^a } \right)^b = x^{\left( {ab} \right)}$\\
$x^{\left( {a - b} \right)} = \frac{{x^a }}{{x^b }} $\\
\end{multicols*}
\section{Common Infinite Series}
\textbf{Exponential}\\
$\sum_{k=0}^{\infty}\frac{z^k}{k!}=e^z$\\
$\sum_{k=0}^{\infty}k\frac{z^k}{k!}=ze^z$ Mean of the Poisson\\
$\sum_{k=0}^{\infty}k^2\frac{z^k}{k!}=(z+z^2)e^z$ Second moment of Poisson\\
$\sum_{k=0}^{\infty}k^3\frac{z^k}{k!}=(z+3z^2+z^3)e^z$\\
$\sum_{k=0}^{\infty}k^4\frac{z^k}{k!}=(z+7z^2+6z^3+z^4)e^z$ \\
\textbf{Binomial}\\
$\sum_{k=0}^{\infty}{\alpha \choose k}z^k=(1+z)^\alpha, |z|<1$\\
$\sum_{k=0}^{n}{n \choose k}x^{n-k}y^k=(x+y)^n$\\
\section{Integration}
\begin{multicols*}{2}
$\int x^n dx = \frac{1}{n+1}x^{n+1}, \hspace{1ex} n\neq-1$

$\int \frac{1}{x} dx = \ln |x|$

$\int u \hspace{2pt} dv = uv - \int v du$

$\int e^x dx = e^x $

$\int a^x dx = \frac{1}{\ln a} a^x$

$\int \ln x dx = x \ln x - x$

$\int \sin x dx = -\cos x$

$\int \cos x dx = \sin x$

\end{multicols*}

\section{The Bernoulli Schema}
\textbf{Theorem 1} \textit{Suppose $X_1,...,X_n$ is a sequence of Bernoulli tials with success probability p. Then the sume $S_n=X_1+...+X_n$ has a distribution $b_n(k)={n \choose k }p^kq^{n-k}$. Recall as a matter of convention the definition for real $t$ and integer $k$ that ${t \choose k} =t(t-1)(t-2)...(t-k+1)/k!$ if $k\geq 0$ and $0$ otherwise.}
\textbf{Example: Ball and Urn} \textit{Put n balls in m urns, probability that there are k balls in first r: ${n \choose k } \frac{r}{m}^k(1-\frac{r}{m})^{n-k}$}\\
\textbf{Example: Polls} \textit{Select n individuals from population, an unknown fraction p support policy. All samples are independent.$S_n=\sum_{j=1}^{n}=Z_j$ is the number of individuals in favor, $S_n \sim b_n(k;p)$. Let $\zeta \in [0,1]$ be a guess for $p$. $b(\zeta)=b_n(S_n;\zeta)={n \choose S_n}\zeta^{S_n}(1-\zeta)^{n-S_n}$, then $b(\zeta)$ is maximized at $\zeta=(S_n/n)$. The sample mean is $\arg max_{\zeta}b_n(S_n);\zeta)$ which is the mean of the empirical distribution. Thus, p  estimated by this value yields the largest probability of observations consistent with the data. We say that the sample mean is the unbiased estimator of p}\\
\textbf{Error Bound on Sample Mean} \textit{Via Chebyshev $P\{|\frac{1}{n}S_n -p|\geq\epsilon\}=\sum b_n(k;p) \leq \frac{pq}{n\epsilon^2}\leq \frac{1}{4n\epsilon^2}$}\\
\textbf{Example: Random Walks} \textit{The number of paths from (a,b) to (a',b') is $N_n(k)={n \choose \frac{n+k}{2}}$ where $n=a'-a$ and $k=b'-b$. If the path goes through strictly positive values the number of such paths is $N^{+}_{n}(k)=N_{n-1}(k-1)-N_{n-1}(k+1) = {n-1 \choose \frac{n+k}{2}-1}-{n-1 \choose \frac{n+k}{2}}=\frac{k}{n}{n \choose \frac{n+k}{2}}$}\\
\textbf{Example: Returns} \textit{The probability of a return is $ P\{S_{2}=N_{2v}(0)2^{-2v}={2v \choose v}2^{-2v}$}\\
\textbf{Example: Waiting Time} \textit{The waiting time is the wait until the first succes $w(k;p)=P(W=k)=q^kp (k\geq 0)$. Now $\sum_{k}w(k;p)=p \sum_{k=0}^{\infty}q^k=p/(1-q)=1$. Thus $w(k;p)$ is called the geometric distribution. Useful manipulation: $\sum_{k}^{\infty}q^k=1/(1-q)$}\\
\textbf{Sum of Independent RVs ~ geometric}\\
\textit{Suppose $\{W_i, i\geq 1\}$ is a sequence of independent geometric random variables with common success parameter $p>0$. For each r, let $T_r=W_1+W_2+...+W_r$ and let $w_r(k;p) =P(T_r=k)$ be its distribution. Then $w_r(k;p)={r+k-1 \choose k} q^kp^r$ this is the negative binomial.}\\
\textbf{Problem of the Points} \textit{ $P_m,n(p)=\sum_{k=0}^{n-1}{-m \choose k}(-q)^kp^m$ where m is the require points to win and n is the maximum number of failures}\\
\textbf{Example: Population Size} \textit{Hypergeometric Distribution: $h(i)=h_{m,n}(i;r)=\frac{{m \choose i}{n-m \choose r-i}}{{n \choose r}}$ where $(0 \leq u \leq r)$}\\
\textbf{Vandermonde's Convolution} \textit{ $\sum_{i} {m \choose i} {n-m \choose r-i} = {n \choose r}$ can be used to show hypergeometric is a density}\\
\textbf{Another funny sum} \textit{${n \choose 0}^2 +{n \choose 1}^2 +...+ {n \choose n}^2={2n \choose n}$}\\

\section{The Essence of Randomness}
\textbf{Buffon's Needle}\\
\textbf{Gamma Function etc} \textit{Continuous version $g_v(x,\alpha)=\frac{\alpha}{\Gamma(v)}(\alpha x)^{v-1}e^{-\alpha x}$}
\section{The Coda of the Normal}
\textbf{Bivariate Normal}\\
\textit{Suppose $(X_1,X_2)$ is a random pair governed by the bivariate normal density $\phi(x_1,x_2,\rho)$ and $Y=X_1+X_2$ then $X_1$ and $X_2$ share a common marginal normal distribution with zero mean and unit variance. Moreover, $Cov(X_1,X_2)=E(X_1 X_2)=\rho$ By additivity of expectation $E(Y)= E(X_1)+E(X_2)=0$, while, $Var(Y)=E((X_1+X_2)^2)=E(X_1^2)+E(X_2^2)+2E(X_1 X_2)=2(1+\rho)$}

\section{Random}
\textbf{The Hockey Stick Identity}
\textit{$\sum_{j=k}^{n}{j \choose n} = {n+1 \choose k+1}$}
\section{Convergence}
\textit{A sequence of random variables $\{X_1,X_2,...\}$ converges almost surely to a random variable $X$ if $P\{s\in S:lim_{n\rightarrow \infty}X_n(s)=X_s\}=1$}
\end{multicols*}
\end{document}
